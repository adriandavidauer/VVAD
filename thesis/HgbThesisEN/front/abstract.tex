\chapter{Abstract}
Technology is integrating more and more into the life of the modern man. A very important
question is how are people interacting with technology.
The human brain does not react emotionally to artificial objects like computers and
mobile phones. However, the human brain reacts strongly to human appearances like
the shape of the human body or faces. Therefore humanoid robots are the most
natural way for human-machine interaction, because of the human-like appearance.
It is obvious that not only the physical appearance needs to be human-like, also the psychological concepts of interaction need to be implemented.
Modern social robots like Pepper from SoftBank Robotics Group Corp. already come with a handful of cognitive features like stimuli reaction or gaze detection.

To make human-machine interaction more natural further cognitive features need to be implemented. 
For this purpose, this proposed master thesis will deal with the problem of a Visual Voice Activity Detection (VVAD), which can detect whether a person is speaking to a robot or not, given the visual input of the robot's camera.
This cognitive feature can solve the problem of
speaker identification in situations, where it is not trivial like crowded areas and when more than one person is in the robot's field of view.

To implement the described VVAD an approach based on a recurrent neural network, to model the time dependencies in the input data,
and convolutional layers, to learn features from pixels, is presented.
Most classic approaches to VVAD unfold to lip motion detection which yields in a high false positive rate because humans make arbitrary lip motion even when not speaking.
To deal with this the network is trained on a slightly adjusted version of the LRS3 dataset. 
The LRS3 dataset contains more than 100.000 video sequences of people speaking. 
Faces will be extracted from the video sequences using correlation based tracking and face detection. 
The evaluation compares two different approaches.
One extracts facial features from the face detection and uses them as the input vector for the network while the other is End-To-End-Learning which uses the whole image as an input vector of pixels. 
These two approaches can again be gradually divided by the area used as input. 
Either the whole face will be used or only the lips.
We find that for this problem, an LTSM-FCN with images of the whole face works best with 92\% accuracy on the VVAD dataset.
